%!TEX program = xelatex
% 导言区
\documentclass[12pt, letterpaper]{article}
% 文档区
\usepackage{enumerate}
\usepackage{caption}	% 可以设置脚注的各种性质
\usepackage{graphicx}	% 使用图片
\usepackage[xetex]{animate}	% 动图
\usepackage{subfigure}	% 使用子图
\usepackage{wrapfig}	% 使用嵌入的图片
\usepackage{color}	% 字体颜色
\usepackage{xcolor}
\usepackage{fullpage}	% 顾名思义
\usepackage{ctex}	% 中文支持
\usepackage{ulem}	% 字体删除线
\usepackage{amsmath}	% 公式换行
\usepackage[colorlinks,linkcolor=blue]{hyperref}	% 插入超链接
\usepackage{mathrsfs}  	% 花体
\usepackage{listings}	% 代码渲染

\lstdefinestyle{Python}{
    language        = Python,
    frame           = lines, 
    basicstyle      = \footnotesize,
    keywordstyle    = \color{blue},
    stringstyle     = \color{green},
    commentstyle    = \color{red}\ttfamily
}


\title{Homework1}

% 正文区
\begin{document}
% 标题
\maketitle
\section{1. Generative and Discriminative classifiers: Gaussian Bayes and Logistic
Regression}

\subsection{ Specific Gaussian naive Bayes classifiers and logistic regression}
已知$y$是一个布尔型变量,它服从伯努利分布,即$\pi=P(y=1)$以及$P(Y=0)=1-\pi$。对于一个二分类问题,Logistic regression恰好合适。它的输出在$(0,1)$区间内,大于某一阈值判断为正,反之则为负。

在这种情况下,他们二者的关系有如下推导:

\begin{equation}
\begin{aligned}
P(y=1|\mathbf{x})=&\frac{P(\mathbf{x}|y=1)P(y=1)}{P(\mathbf{x})}\\
=&\frac{P(\mathbf{x}|y=1)P(y=1)}{P(\mathbf{x}|y=0)P(y=0)+P(\mathbf{x}|y=1)P(y=1)}
\label{a}
\end{aligned}
\end{equation}
已知向量空间为$\mathbf{x}$为$\mathbf{x}=[x_1,\ldots,x_D]$,且对于每一个特征$x_i$,它都符合如下高斯分布:$\mathcal{N}(u_{ik},\sigma_i)$,其中$\sigma_i$是高斯分布的标准差,它独立于$k$。把这些条件带入\ref{a}可得:
\begin{equation}
\begin{aligned}
P(y=1|\mathbf{x})&=\frac{\pi\prod_{i=1}^D\mathcal{N}(u_{i1},\sigma_i)}{(1-\pi)\prod_{i=1}^D\mathcal{N}(u_{i0},\sigma_i)+\pi\prod_{i=1}^D\mathcal{N}(u_{i1},\sigma_i)}\\
&=\frac{1}{1+\frac{1-\pi}{\pi}\frac{\prod_{i=1}^D\mathcal{N}(u_{i0},\sigma_i)}{\prod_{i=1}^D\mathcal{N}(u_{i1},\sigma_i)}}\\
&=\frac{1}{1+\frac{1-\pi}{\pi}\frac{e^{-\sum_{i=1}^D\frac{(x_i-u_{i0})^2}{2\sigma_i^2}}\prod_{i=1}^D(\frac{1}{\sqrt{2\pi}\sigma_i})}{e^{-\sum_{i=1}^D\frac{(x_i-u_{i1})^2}{2\sigma_i^2}}\prod_{i=1}^D(\frac{1}{\sqrt{2\pi}\sigma_i})}}\\
&=\frac{1}{1+\frac{1-\pi}{\pi}e^{\sum_{i=1}^D\frac{(x_i-u_{i1})^2-(x_i-u_{i0})^2}{2\sigma_i^2}}}\\
&=\frac{1}{1+e^{(\ln\frac{1-\pi}{\pi}+\sum_{i=1}^D[\frac{(u_{i0}-u_{i1})x_i}{\sigma^2}+\frac{u_{i1}^2-u_{i0}^2}{2\sigma_i^2}])}}\\
&=\frac{1}{1+e^{(w_0+\sum_{i=1}^Dw_ix_i)}}
\label{b}
\end{aligned}
\end{equation}

\subsection{General Gaussian naive Bayes classifiers and logistic regression}
将$\mathcal{N}(u_{ik},\sigma_{ik})$带入\ref{a}中可以得到下式:


\begin{equation}
\begin{aligned}
P(y=1|\mathbf{x})&=\frac{\pi\prod_{i=1}^D\mathcal{N}(u_{i1},\sigma_{i1})}{(1-\pi)\prod_{i=1}^D\mathcal{N}(u_{i0},\sigma_{i0})+\pi\prod_{i=1}^D\mathcal{N}(u_{i1},\sigma_{i1})}\\
&=\frac{1}{1+\frac{1-\pi}{\pi}\frac{\prod_{i=1}^D\mathcal{N}(u_{i0},\sigma_{i0})}{\prod_{i=1}^D\mathcal{N}(u_{i1},\sigma_{i1})}}\\
&=\frac{1}{1+\frac{1-\pi}{\pi}\frac{e^{-\sum_{i=1}^D\frac{(x_i-u_{i0})^2}{2\sigma_{i0}^2}}\prod_{i=1}^D(\frac{1}{\sqrt{2\pi}\sigma_{i0}})}{e^{-\sum_{i=1}^D\frac{(x_i-u_{i1})^2}{2\sigma_{i1}^2}}\prod_{i=1}^D(\frac{1}{\sqrt{2\pi}\sigma_{i1}})}}\\
&=\frac{1}{1+e^{(\ln{\frac{1-\pi}{\pi}+\sum_{i=1}^D\frac{\sigma_{i1}}{\sigma_{i0}}+\sum_{i=1}^D[\frac{(x_i-u_{i1})^2}{2\sigma^2_{i1}}-\frac{(x_i-u_{i0})^2}{2\sigma_{i0}^2}}])}}
\end{aligned}
\end{equation}

由于$\sigma_{i0}$和$\sigma_{i1}$不相同,因此上式不能化为\ref{b}的形式。即$e$的指数无法化为$w_o+\sum_{i=1}^Dw_ix_i$的格式。

\subsection{Gaussian Bayes classifiers and logistic regression}
对于特征$x_i,x_j(i\neq j)$不相互独立的情况下。有:
\begin{equation}
\begin{aligned}
P(y=1|\mathbf{x})&=\frac{1}{1+\frac{P(\mathbf{x}|y=0)P(y=0)}{P(\mathbf{x}|y=1)P(y=1)}}\\
&=\frac{1}{1+\frac{1-\pi}{\pi}\frac{P(x_1,x_2|y=0)}{P(x_1,x_2|y=1)}}\\
\label{c}
\end{aligned}
\end{equation}
将$P(x_1,x_2|y=k)$的公式带入\ref{c}可得:
\begin{equation}
\begin{aligned}
P(y=1|\mathbf{x})&=\frac{1}{1+\frac{1-\pi}{\pi}e^{M(x_1,x_2)}}
\label{d}
\end{aligned}
\end{equation}
其中,$M(x_1,x_2)$为:
$$
M(x_1,x_2)=
\frac{N_1(x_1)+N_2(x_2)+N_{1,2}(x_1,x_2)}{2(1-\rho^2)\sigma_1^2\sigma_2^2}
$$
其中:

$$
N_1(x_1)=\sigma_2^2(u_{11}-u{10})[u_{11}+u_{10}-2x_1]
$$
$$
N_2(x_2)=\sigma_1^2(u_{21}-u{20})[u_{21}-u_{20}-2x_2]
$$
$$
N_{1,2}(x_1,x_2)=2\rho\sigma_1\sigma_2[(u_{21}-u_{20})x_1+(u_{11}-u_{10})x_2+u_{10}u_{20}-u_{11}u_{21}]
$$

显然,\ref{d}中的变量$x_1,x_2$都是一次的,那么\ref{d}式显然可以被整理为\ref{b}的结果:
\begin{equation}
P(y=1|\mathbf{x})=\frac{1}{1+e^{w_1x_1+w_2x_2+w_0}}
\end{equation}



















\end{document}