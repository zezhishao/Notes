%!TEX program = xelatex
% 导言区
\documentclass[12pt, letterpaper]{article}
% 文档区
\usepackage{caption}	% 可以设置脚注的各种性质
\usepackage{graphicx}	% 使用图片
\usepackage{subfigure}	% 使用子图
\usepackage{wrapfig}	% 使用嵌入的图片
\usepackage{color}	% 字体颜色
\usepackage{xcolor}
\usepackage{fullpage}	% 顾名思义
\usepackage{ctex}	% 中文支持
\usepackage{ulem}	% 字体删除线
\usepackage{amsmath}	% 公式换行
\usepackage[colorlinks,linkcolor=blue]{hyperref}	% 插入超链接
\usepackage{mathrsfs}  	% 花体
\usepackage{showlabels}

\title{一些想法}

% 正文区
\begin{document}
% 标题
\maketitle
% 目录
\tableofcontents
\newpage

\section{数据和模型}\label{aaaa}
机器学习中,数据可以被分为四大类:
\begin{itemize}
\item 图像:Image
\item 序列:Sequence
\item 图:Graph
\item 表格:Tabular
\end{itemize}
其中,前三类有着比较明显的模式。比如\textbf{图像和图的空间局部性},\textbf{序列的上下文关系和时序依赖性}。而表格数据常见于\textbf{各种工业界的任务},比如广告点击率预测,推荐系统等。在表格数据中,每一个特征表示一个属性,如性别、价格,特征之间一般没有明显且通用的模式。

神经网络适合前三类数据,也就是有明显模式的数据。\textbf{针对不同的数据模式,设计对应的网络结构},从而实现高效地自动抽取“高级”的特征表达。如CNN(图像)、RNN(序列)。\textbf{而表格数据,没有明显的模式。}神经网络无法针对设计。因此对于表格数据,除了专门对特定的任务设计的网络结构如DeepFM等,更多时候还是用传统的机器学习模型。尤其是LGBT(梯度提升术)。因其自动的特征选择能力及动态的模型复杂度,算得上是一个万金油模型,在各种类型的表格数据上都表现很好。

对于表格数据而言,特征工程更加关键,在给定数据的情况下,模型决定了下限,特征决定上限。\textbf{特征工程类似于神经网络的结构设计,目的是把先验知识融入数据。}

No free lunch,没有万能的模型。用神经网络,需要结构设计;使用传统模型,需要特征工程。

\section{基本的机器学习方法}
机器学习的基础方法大概有六种:K近邻算法、主成分分析法、逻辑回归算法、朴素贝叶斯分类器、决策树算法、随机森林、支持向量机算法、K-Means聚类、人工神经网络ANN。\textbf{每种方法都有其适应的场景、对象,以及其内涵}


\section{机器学习的分类}
机器学习分为有监督和无监督两种。

对于有监督学习,是给定数据集$\{x^i, y^i\}$,学习出$\hat{y}=f(x)$。\textbf{对于分类问题},y是\textbf{类别}。\textbf{对于回归问题},y是\textbf{连续数};\textbf{对于排序问题}(尤其是信息检索和网页排序等应用上),y是\textbf{序值}。一般情况下,有监督学习就分为这几种。每一个有监督学习都可以归结到这几类问题中,并对照
数据和模型\ref{aaaa}选择合适的方法。


无监督学习是给定数据集$\{x^i\}$,学习出$\hat{y}=f(x)$。\textbf{对于密度估计},y是\textbf{密度}。\textbf{对于聚类},y是\textbf{类簇}。\textbf{对于数据规约、数据可视化},y是\textbf{数据x的低纬表示}(例如Autoencoder)。无监督想是发展的方向,当前无监督学习经常被作为监督学习的预处理步骤(这是当前监督学习的有种流行的范式)。

\end{document}